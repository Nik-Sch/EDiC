% !TEX root = ../thesis.tex
\chapter{Introduction} \label{cha:intro}
The foundation of this thesis started at the end of 2020 where I decided to design and build a \gls{CPU} from scratch.
In many university courses we would discuss some parts of a \gls{CPU} like different approaches to binary adders or pipelining concepts but never would we build a complete \gls{CPU} including the control logic.
Due to a Covid-19 lockdown I had enough time at my hands and after 6 years of study, I felt like I had the expertise to complete this project.

\begin{figure}[t]
  \centering
  \includegraphics[width=\textwidth]{update20210121_crop.jpg}
  \caption{The first version of the \gls{CPU} in its final state.}
  \label{fig:initialCPU}
\end{figure}
At the end of January 2021 I succeeded with the actual hardware built and the \gls{CPU} was able to execute a prime factorization of 7 bit numbers.
\Cref{fig:initialCPU} depicts the final hardware build.
Its design ideas, implementation and flaws are shown in \cref{cha:prev}.

Through the university module ``Mixed-Signal-Baugruppen'' I got to know Henry Westphal in summer 2021.
He established a company that builds mixed-signal-electronics and, therefore, has a deep understanding of analog and digital circuitry.
As he heard of my plans to build a future version of my \gls{CPU} he was immediately interested and we wanted to rebuild a \gls{CPU} with some changes:
\begin{itemize}
  \item The general architecture should remain similar to the existing \gls{CPU} with only changes where it was necessary.
  \item The objective was no longer only to create a functioning \gls{CPU}, this was already accomplished, but the build should be such that it could be used for education.
  \item It should be more reliable, more capable and its components should be easily distinguishable. Therefore, it is to be build on a large \gls{PCB}.
  \item There should be a generic interface for extension cards, i.e. IO Devices.
\end{itemize}
How the, now called, \gls{EDiC} differs from its predecessor is presented in \cref{cha:designChanges}.

To achieve the goal of the \gls{EDiC} being educational it is important to not only build the hardware but to also provide a Software Environment to, for example, write applications.
This is presented in \cref{cha:software}.

An important step in the design of the \gls{EDiC} was to thoroughly simulate and implement the behavior on an \gls{FPGA}.
I firstly simulated the behavior and after the hardware schematic was finished, we built a script to convert the exported netlist to verilog to simulate the \gls{CPU} on chip level.
The process and differences between the \gls{FPGA} design and the actual hardware are presented in \cref{cha:fpga}.

\Cref{cha:hardware} describes the final hardware assembly, commissioning and timing analysis to determine the final clock frequency.

The final conclusion and future improvements are given in \cref{cha:conclusion}.