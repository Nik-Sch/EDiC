% !TEX root = ../thesis.tex
\chapter{FPGA Model}\label{cha:fpga}
The goal of the \gls{FPGA} simulation is to proof the general workings of the \gls{CPU} architecture.
There was no attempt made to provide a chip-level simulation of the hardware build but rather to provide a top-level behavioral model.
The chosen development environment is the AMD - Xilinx Vivado \cite{vivado} as it is freely available and provides an advanced simulation environment while providing the possibility to synthesize for relatively cheap \glspl{FPGA}.

One major problem with tri-state bus logic for \glspl{FPGA} is that most current era \glspl{FPGA} do not feature tri-state bus drivers in the logic.
Most \glspl{FPGA} do have bidirectional tri-state transceiver for I/O logic but not for internal logic routing.
However, the \glspl{HDL} (both \gls{VHDL} and Verilog) support tri-state logic and the Xilinx Simulation tool also does.
As the first \gls{CPU} was only simulated, this was not a problem and the tri-state logic could be used the same way as in the hardware build.
\Cref{cha:fpga} describes how tri-state logic is solved for the synthesis of the \gls{EDiC}.

\subsection{Language Choice}
There are two main \glspl{HDL}: Verilog and \gls{VHDL}.
Both are widely supported and used and can also be used in parallel in the same design.
At the \gls{TUB} \gls{VHDL} is taught and in Germany it also used more often.
However, in general both are used about equally often \cite{vhdlVerilog}.

As I only knew \gls{VHDL} and very basic concepts of Verilog, I decided to start in Verilog to get to know the differences.
\begin{listing}
  \inputminted[linenos,
    breaklines,
    firstline=24,
    lastline=61,
    % firstnumber=1,
    frame=leftline,
    xleftmargin=20pt,
  ]{verilog}{src/alu_proto.sv}
  \caption{(System)-Verilog Code for the \gls{ALU} of the first \gls{CPU} version.}
  \label{lst:alu_proto}
\end{listing}
\Cref{lst:alu_proto} shows the Verilog Code for the \gls{ALU} module (without the module definition to fit on one page).
Lines 24-28 describe the synchronous, positive-edge-triggered alu output register with a write enable.
The 8 XOR for the B input are described by lines 39-42 and lines 44-61 show an combinatorial process for the alu operation and multiplexing.
The lines 50-58 describe the bidirectional barrel shifter with 3 shift steps (by 1, 2, and 4) and the reverse \glspl{MUX} in front and at the end.

\section{Behavioral Simulation}
\section{Behavioral Implementation}
\section{Chip-level Implementation}
\subsection{Conversation Script}