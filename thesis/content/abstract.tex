% !TeX root = ../thesis.tex
\chapter*{Abstract}
This thesis covers the implementation of the \gls{EDiC}, a model \gls{CPU} which is to be used for teaching the workings of modern digital general purpose processors.
For educational purposes the novel \gls{CPU} \gls{ISA} is accompanied by an extensive software development environment.
The thesis justifies the architectural design decisions which lead to the creation of this 8 bit \gls{CISC} multi-cycle \gls{CPU} with a 16 bit address space and comprehensive \gls{IO} support.
The breakdown of the \gls{CPU} into seven independent modules simplifies the process of understanding the details of the \gls{CPU}.
Additionally, the choice of \gls{TTL} \glspl{IC} of the 74 family takes the learning focus towards the digital level without complicating the design with analog behavior as \gls{RTL} would.

For the functional verification, a behavioral as well as a chip-level \gls{FPGA} implementation is performed.
The component verification is eased with specially developed test adapters which allow for bit by bit testing of all \glspl{IC} and in-depth debugging.
With a detailed timing analysis it is ensured that the \gls{EDiC} does not run into unpredictable timing problems.
\begingroup
\renewcommand{\cleardoublepage}{}
\clearpage
\chapter*{Kurzfassung}
\endgroup
\glsresetall
Diese Arbeit beschreibt die Entwicklung und Implementierung des \gls{EDiC}, einer Model \gls{CPU}, welche speziell für die Lehre entwickelt wurde.
Sie soll dabei helfen, die Funktionsweise eines modernen, allgemein benutzbaren Prozessors zu vermitteln.
Dafür wird die neu entwickelte Befehlssatzarchitektur (engl. ISA) durch eine ausführliche Entwicklungsumgebung unterstützt.
Alle Designentscheidungen, welche zu diesem 8 bit \gls{CISC} mit einem 16 bit Adressraum beigetragen haben, werden ausführlich erklärt und abgewogen.
Die Aufteilung in insgesamt sieben größtenteils unabhängige Module vereinfacht das Verständnis der Details der \gls{CPU} deutlich.
Das Verständnis der \gls{CPU} wird zusätzlich durch Wahl der Integrierten Schaltkreise (engl. IC) mit Transistor-Transistor-Logik (TTL) aus der 74er Familie deutlich vereinfacht, weil hier der Fokus auf der digitalen Ebene liegt und der Betrachter sich nicht, wie zum Beispiel bei Widerstands-Transistor-Logik (engl. RTL), mut analogen Nebeneffekte beschäftigen muss.

Um die Funktionalität zu verifizieren, wurden zwei \gls{FPGA}-Implementierungen durchgeführt.
Die erste Implementierung, modeliert nur das Verhalten der Schaltung, während die zweite Implementierung auf \gls{IC}-level den Schaltplan verifiziert.
Die Verifikation der einzelnen Hardware-Komponenten wird durch speziell für den \gls{EDiC} entwickelte Test Adapter deutlich vereinfacht.
Diese ermöglichen es, all \glspl{IC} Bit für Bit zu testen und die Schaltung ausführlich zu debuggen.
Weiterhin wird durch eine detaillierte Timing-Analyse sichergestellt, dass beim \gls{EDiC} keine unvorhergesehenen Timing-Probleme auftreten werden.