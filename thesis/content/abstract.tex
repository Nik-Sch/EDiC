% !TeX root = ../thesis.tex
\chapter*{Abstract}
This thesis covers the implementation of the \gls{EDiC}, a model \gls{CPU} which is to be used for teaching the workings of modern digital general purpose processors.
For the educational purposes an extensive software development environment accompanies the novel \gls{CPU} \gls{ISA}.
The thesis justifies the architectural design decisions which lead to the design of this 8 bit \gls{CISC} multi-cycle \gls{CPU} with a 16 bit address space and comprehensive \gls{IO} support.
The modularization of the \gls{CPU} into 7 independent modules simplifies the process of understanding the details of the \gls{CPU}.
Additionally, the choice of \gls{TTL} \glspl{IC} of the 74 family takes the learning focus towards the digital level without complicating the design with analog behavior as \gls{RTL} would.

For functional verification, a behavioral and also a chip-level \gls{FPGA} implementation is performed.
The component verification is eased with specially developed test adapter which allow for bit by bit testing of all \glspl{IC} and in-depth debugging.
With a detailed timing analysis it is ensured that the \gls{EDiC} does not run into unpredictable timing problems.
\begingroup
\renewcommand{\cleardoublepage}{}
\clearpage
\chapter*{Kurzfassung}
\endgroup
\glsresetall
Diese Arbeit beschreibt die Entwicklung und Implementierung vom \gls{EDiC}, einer Model \gls{CPU} welche speziell für die Lehre entwickelt wurde.
Sie soll dabei helfen die Funktionsweise eines modernen, allgemein benutzbaren Prozessors erklären.
Dafür wird die neu entwickelte \gls{ISA} durch eine ausführliche Entwicklungsumgebung unterstützt.
Alle Design Entscheidungen, welche zu dieser 8 bit \gls{CISC} und multi-cycle \gls{CPU} mit einem 16 bit Adressraum geführt haben, werden ausführlich erklärt und abgewogen.
Die Aufteilung in insgesamt 7 größtenteils unabhängige Module vereinfacht das Verständnis der Details der \gls{CPU} enorm.
Zusätzlich wird das Verständnis durch die Wahl der \gls{TTL} \glspl{IC} aus der 74er Familie auf die digitale Ebene gelenkt und nicht durch analoge Nebeneffekte wie bei \gls{RTL} abgelenkt.

Um die Funktionalität zu verifizieren wurde eine Verhaltensimplementierung und auch eine Implementierung auf \gls{IC}-level auf einem \gls{FPGA} durchgeführt.
The Verifikation der einzelnen Hardware Komponenten wird durch speziell für den \gls{EDiC} designte Test Adapter deutlich vereinfacht.
Diese erlauben ein Bit für Bit testen von allen \glspl{IC} und ausführliches debuggen der Schaltung.
Weiterhin wird durch eine detaillierte Timing-Analyse sichergestellt, dass beim \gls{EDiC} keine unvorhergesehenen Timing-Probleme auftreten werden.